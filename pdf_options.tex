%% Basierend auf einer TeXnicCenter-Vorlage von Tino Weinkauf.

%%%%%%%%%%%%%%%%%%%%%%%%%%%%%%%%%%%%%%%%%%%%%%%%%%%%%%%%%%%%%%


%%%%%%%%%%%%%%%%%%%%%%%%%%%%%%%%%%%%%%%%%%%%%%%%%%%%%%%%%%%%%

%% PDF-Informationen

%%%%%%%%%%%%%%%%%%%%%%%%%%%%%%%%%%%%%%%%%%%%%%%%%%%%%%%%%%%%%

%%

%% ACHTUNG: Sie ben�tigen ein Hauptdokument, um diese Datei

%%          benutzen zu k�nnen. Verwenden Sie im Hauptdokument

%%          den Befehl "\input{dateiname}", um diese

%%          Datei einzubinden.

%%


\pdfinfo{                               % Zusatzinformationen in PDF-Datei;

                                        % alle Werte sind optional.

    /Author (Ronald Becher)

    /CreationDate (D:20090928120000)    % Datum der Erstellung

                                        % (D:JJJJMMTThhmmss)

                                        % JJJJ  Jahr

                                        % MM    Monat

                                        % TT    Tag

                                        % hh    Stunden

                                        % mm    Minuten

                                        % ss    Sekunden

                                        %

                                        % Standard: Das aktuelle Datum

                                        %

    /ModDate (\date)         % Datum der letzten Modifikation

    /Creator (TeX && TXC)               % Standard: "TeX"

    /Producer (pdfTeX)                  % Standard: "pdfTeX" + pdftex version

    /Title (Vorlesungsmitschrift Stochastik Wintersemester 2009)

    /Subject (Thema Ihres Dokumentes)

    /Keywords (stochastik, mathematik, vorlesung, mitschrift)

}

