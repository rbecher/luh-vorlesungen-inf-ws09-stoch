Sei $(\Omega,\mathfrak{B},P)$ der W-Raum unserer Wahl. P habe Dichte $f$, also eine Teilfunktion $f:R\rightarrow$, $\int_{- \infty}^{+ \infty}{f(t)dt)}=1$. $F(x)=P((-\infty,x])=\int_{- \infty}^{+ \infty}{f(t)dt)},\ x \in R$.

Allgemein gilt $P(E)=\int_{E}{f(t)dt)},\ E \in \mathfrak{B} (geeignet)$.

Beispiel: 
\[ f(t) = \begin{cases}
0 & t < 0\\
\lambda\cdot exp(-\lambda t) & t \geq 0
\end{cases}
\]

% Graph dazu

Aus

\[
	\int_0^x{\lambda \cdot exp(-\lambda t)dt} = \left[- exp(-\lambda t)\right]_o^x = 1-exp(-\lambda x) % aufleitung eigtl. mit unter und �ber ...
\]

folgt

\[ F(x) = \int_{- \infty}^{+ \infty}{f(t)dt}=\begin{cases}
0 & x < 0\\
1 - exp(-\lambda x) & x \geq 0
\end{cases}
\]

% Graph dazu

\section{Weitere Eigenschaften von Wahrscheinlichkeitsma�en}

%% TODO Suche und ersetze R durch \Oemga in W-Ma0en
Sei $(R,\mathfrak{A},P)$ W-Raum. Seien $A_1,A_2,\ldots \in \mathfrak{A}$. Dann gilt
\begin{equation}
	P\left( \bigcup_{n=1}^{\infty}A_n \right) \leq \sum_{n=1}^{\infty}{P(A_n)}
	\label{eq:subsigmaadd}
\end{equation}
Diese Eigenschaft wird als \textbf{Sub-$\sigma$-Additivit�t} bezeichnet.

\begin{equation}
	\bigcup_{n=1}^{\infty}A_n = A_1 + \sum_{n=2}^{\infty}{(A_n \cap A_1^c\cap\ldots\cap A_{n-1}^c)}
	\label{eq:subsigmadisj}
\end{equation}
Wir k�nnen die Vereinigung auch als Summe von disjunkten Ereignissen auffassen und gelangen zu dieser Schreibweise in \eqref{eq:subsigmaadd}. Deshalb gilt auch \eqref{eq:subsigmadisj}

\begin{equation}
	\begin{split}
				& P\left( \bigcup_{n=1}^{\infty}A_n \right) \\
	=			& \sum P(A_1) + \sum_{n=2}^{\infty}{P(A_n \cap A_1^c\cap\ldots\cap A_{n-1}^c)}\\ % klammer unter mengeninhalt \leq P(A_n)
	\leq 	&  \sum_{n=1}^{\infty}{P(A_n)}
	\end{split}
\end{equation}

Speziell folgt hieraus die so genannte ``Sub-Additivit�t von P'':
\begin{equation}
	P\left( \bigcup_{j=1}^{n}A_j \right) \leq \sum_{j=1}^{n}{P(A_j)}
\end{equation}

Seien $A_n \in \mathfrak{A}, n = 1,2,\ldots, A \in \mathfrak{A}$. Wir schreiben 
\begin{itemize}
	\item $A_n\!\uparrow\! A$ f�r $A_1 \subset A_2 \subset \ldots, A = P\left( \bigcup_{j=1}^{n}A_j \right)$\\
				Man sagt ``$A_n$ konvergiert \textbf{isoton} gegen A'')
	\item $A_n\!\downarrow\! A$ f�r $A_1 \subset A_2 \subset \ldots, A = P\left( \bigcap_{j=1}^{n}A_j \right)$\\
				Man sagt ``$A_n$ konvergiert \textbf{antiton} gegen A'')
\end{itemize}

\section{Stetigkeitseigenschaften von W-Ma�en}
\begin{enumerate}
	\item $A_n\! \uparrow\! A \Longrightarrow \lim_{n\rightarrow\infty}{P(A_n)=P(A)}$
	\item $A_n\! \downarrow\! A \Longrightarrow \lim_{n\rightarrow\infty}{P(A_n)=P(A)}$
\end{enumerate}

\subsection{Beweis zu (1)}

\begin{equation}
	A = \bigcup_{n=1}^{\infty}{A_n} = A_1 + \sum_{n=2}^{\infty}{(A_n \cap A_{n-1}^c)}
\end{equation}

Also

\begin{equation}
	\begin{split}
	P(A) 	&= P(A) + \sum_{n=2}^{\infty}{P(A_n \cap A_{n-1}^c)}\\ % in der Menge ist P(A_n) - P(A_n-1), hei�t Subtraktivit�t von P 
				&= P(A) + \sum_{n=2}^{\infty}{P(A_n - A_{n-1}^c)}\\
				&= P(A) + \lim_{m\rightarrow\infty}{\sum_{n=2}^{m}{P(A_n - A_{n-1})}}\\
				&= P(A_1) + \lim_{m\rightarrow\infty}{(P(A_m - A_{1}))}\\
				&= \lim_{m\rightarrow\infty}{P(A_m)}
	\end{split}
\end{equation}

\subsection{Beweis zu (2)}

Mit Gegenmenge ... 

\subsection{Beispiel}

$\Omega=R, x\in R, (x_n)\subset R$ Folge, die monoton fallend gegen $x$ konvergiert.
\[
	(-\infty,x] = \bigcap_{n=1}^{\infty}(-\infty,x_n]
\]

$P$ ist W-Ma� auf $R$
\[
	F(x) = P((-\infty,x])=\lim_{n\rightarrow\infty}P((-\infty,x_n]) = \lim_{n\rightarrow}F(x_n) % Hinweis: F(x) ist Verteilungsfkt. von P
\]

Hieraus folgt allgemein:\\
Ist $(x_n)\subset R$ eine Folge, die von rechts gegen $x$ konvergiert (d.h. $x\leq x_n$ f�r jedes $n \leftarrow N$ und $\lim_{n\rightarrow}x_n=x$), so gilt
\[
	\lim_{n\rightarrow \infty} F(x_n)=F(x)
\]
(``$F$ ist rechtsseitig stetig'')

\newpage

\part{Zufallsvariablen und ihre Verteilung}
\label{part:zufallsvariablen}

Wir gehen von einem Wahrscheinlichkeitsraum $(\Omega,\mathfrak{A},P)$ aus, $R$ sei unsere gesuchte Bild-Menge, $\mathfrak{S}$ $\sigma$-Algebra auf $R$.
\[X:\Omega \rightarrow R\]
sei Wahrscheinlichkeitsfunktion mit der Eigenschaft, dass f�r jedes $B \in \mathfrak{S}$ gilt
\begin{equation}
	X^{-1}(B) = \{ \omega \in \Omega; X(\omega) \in B \} \in \mathfrak{A}
\end{equation}
hei�t Zufallsvariable.

\begin{itemize}
	\item \begin{equation}
	\{ X \in B \} = \{\omega \in \Omega; X(\omega) \in B \} = X^{-1}(B)
\end{equation}
	\item \begin{equation}
				P(X\in B) \text{ f�r } P(\{ X \in B \}) = P(X^{-1}(B))
			\end{equation}
	\item Durch
		\[
			P^X:\mathfrak{S} \rightarrow [0,1]
		\]
		definiert durch
		\[
			P(B) = P(X^{-1}(B)) = P(X \in B),\ B\in \mathfrak{S}
		\]
		wird ein W-Ma� $P^X$ auf $\mathfrak{S}$ definiert.
	\item 
		\[
			X^{-1}\left( \sum_{n=1}^{\infty}{B_n} \right) = \sum_{n=1}^{\infty}{X^{-1}(B_n)}
		\]
\end{itemize}

$P^X$ hei�t die \textbf{Verteilung von $X$}.

\section{Spezialf�lle}
\begin{enumerate}
	\item $R=\mathbb{R}, \mathfrak{S} = \mathfrak{B}$. Eine Zufallsvariable $X:\Omega\rightarrow\mathbb{R}$ hei�t \textbf{reelle Zufallsvariable}.\\
				$F(x)=P^X((-\infty,x]) ) P(X\in ((-\infty,x]))=P(X\leq x), x\in \mathbb{R}$ % letztes = ist "`Schreibweise"'
				ist die Verteilungsfunktion von $P^X$, kurz: Die Verteilungsfunktion von $X$.\\
				Hat $P^X$ die Dichte $f$, so ist
					\[
						F(x) = P(X\leq x) = \int_{-\infty}^{x}{f(t)dt}, x\in\mathbb{R}
					\]
					Sprechweise: $X$ hat die Dichte $f$.
	\item $R \neq \emptyset$, $\mathfrak{S}$ enthalte die Mengen $\{x\}$ f�r $x \in R$.\\
				Es gibt eine abz�hlbare Menge $R_0 \subset R$ mit $P(X \in R_0)=1$.\\
				Dann ist $P(X\not\in R_0)=1-P(X\in R_0)=0$\\
				Es gilt daher
					\[
						P(X \in R_0^c \cap B) = 0,\ \forall B\in \mathfrak{S}
					\]
				und
					\[
						\begin{split}
							P(X\in B)	&= P(X\in R_0 \cap B) + \cancel{P(X \in R_0^c \cap B)}\\
												&= \sum_{x\in R_0\,\cap\,B}{P(X=x)}\\
												&= \sum_{x\in R_0}{P(X=x)\delta_x(B)}
						\end{split}
					\]
				Also
					\[
						P^X = \sum_{x\in R_0}{P(X=x)\delta_x}
					\]
				$X$ hei�t dann diskret verteilt.\\
				$P^X$ ist festgelegt durch die Werte $P(X=x), x\in R_0$
\end{enumerate}

\section{Beispiel: W�rfelwurf von 2 echten W�rfeln}
\begin{itemize}
	\item $\Omega = \{ (\omega_1,\omega_2); \omega_1,\omega_2 \in \{1,\ldots,6\}\}$
	\item $|\Omega| = 6^2 = 36$
	\item $\mathfrak{A} = \mathfrak{P}(\Omega)$
	\item $P(A) = \frac{|A|}{|\Omega|}, A \subset \Omega$
	\item $X:\Omega \rightarrow \mathbb{R}$, $X(\omega_1,\omega_2)=\omega_1+\omega_2, (\omega_1,\omega_2)\in\Omega$
	\item $P(X=k)$
\end{itemize}
Also
\begin{equation}
	\begin{split}
	P(X=k) 	&= P(\{(\omega_1,\omega_2)\in\Omega; \omega_1+\omega_2=k\}) \\
					&= \frac{|\{(\omega_1,\omega_2)\in\Omega; \omega_1+\omega_2=k\}|}{36}\\
					&= \begin{cases}
						 	\frac{1}{36} &, k=2\\
						 	\frac{2}{36} &, k=3\\
						 	\frac{3}{36} &, k=4\\
						 	\vdots &\\
						 	\frac{1}{36} &, k=12\\
						 \end{cases}
	\end{split}
\end{equation}

\noindent Nat�rlich kann man das auch allgemeiner schreiben:\\
Sei $R=\mathbb{R}^d = \mathfrak{B}^d$. $X$ hei�t dann $d$-dimensionaler Zufallsvektor $X:\Omega \rightarrow \mathbb{R}^d$.\\
$X = (X_1,\ldots,X_d)$ mit reellen Zufallsvariablen $X_1,\ldots,X_n$ die Komponenten von $X$. $X$ hei�t nun \textbf{Zufallsvektor}.

\noindent Sei $f:\mathbb{R}^d \rightarrow \mathbb{R}_+$ mit 
\[
	\int_{-\infty}^{+\infty}{}\cdots\int_{-\infty}^{+\infty}{f(t_1,\ldots,t_d)dt_1\cdots dt_d}=1
\]
$f$ hei�t Dichte des Zufallsvektors $X$, wenn gilt
\[
	\begin{split}
	F(x_1,\ldots,x_d) &=P(X_1\leq x_1,\ldots,X_d\leq x_d)\\
										&= \int_{-\infty}^{+\infty}{}\cdots\int_{-\infty}^{+\infty}{f(t_1,\ldots,t_d)dt_1\cdots dt_d},\  (x_1,\ldots,x_d)\in\mathbb{R}^d
	\end{split}
\]
F hei�t Verteilungsfunktion des Zufallsvektors.