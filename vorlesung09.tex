\part{Gesetze der gro�en Zahlen}

\section{Einf�hrung}

Es handelt sich um Grenzwerts�tze der Stochastik. Wir m�ssen uns hierf�r erst einmal Konvergenzbegriffe �berlegen.

Seien $X,X_n,\ n\in \mathbb{N}$ neue, $d$-dim. Zufallsvariablen auf einem W-Raum $(\Omega,\mathfrak{A},P)$.
Die Folge der $X_n$ konvergiert ``$p$-fast-sicher'' gegen $X$, wenn gilt
\begin{equation}
	P\left( \lim_{n\rightarrow\infty}{X_n=X} \right) = P \left( \bigcap_{l\in\mathbb{N}>0}\bigcup_{m\in\mathbb{N}}\bigcap_{n=m}^{\infty}\{|X_n-X| \leq \frac{1}{n}\} \right) = 1
\end{equation}
Hierbei sei $|\cdot |$ die euklidische Norm)

Schreibweise: \[ X_n \rightarrow X,\ P\text{-f.s.} \]
Die Folge der $X_n$ konvergiert stochastisch gegen $X$, wenn gilt
\begin{equation}
	\forall \epsilon>0 : \lim_{n\rightarrow\infty}P\left(|X_n-X| \geq \epsilon\right) = 0
\end{equation}
Schreibweise \[ X_n \stackrel{P}{\rightarrow} X\]

\section{Bemerkungen}

\subsection{a}

\begin{equation}
	\begin{split}
		 & X_n \rightarrow X P\text{-f.s.} &= \iff \text{Es gibt eine P-Nullmenge $N\in \mathfrak{A}$, (d.h. $P(N)=0$) mit der Eig., dass}\\
		 & & \lim_{n\rightarrow\infty}X_n(\omega)=X(\omega) \forall \omega \in N^c\\
		"`\Rightarrow"' & \text{W�hle } N &=\left( \bigcap_{l\in\mathbb{N}>0}\bigcup_{m\in\mathbb{N}}\bigcap_{n=m}^{\infty}\{|X_n-X| \leq \frac{1}{n}\}\right)^c \\
		& & =  \bigcup_{l\in\mathbb{N}>0}\bigcap_{m\in\mathbb{N}}\bigcup_{n=m}^{\infty}\{|X_n-X| \leq \frac{1}{n}\}\\
		"`\Leftarrow"' & N^c & \subset \bigcap_{l\in\mathbb{N}>0}\bigcup_{m\in\mathbb{N}}\bigcap_{n=m}^{\infty}\{|X_n-X| \leq \frac{1}{n}\}
	\end{split}
\end{equation}

\subsection{b}
$X_n \rightarrow X$ $P$-f.s. $\Rightarrow X_n \stackrel{P}{\rightarrow} X$.\\
F�r $l\in\mathbb{N}$ ist
\begin{equation}
	\begin{split}
		 &  \bigcap_{m\in\mathbb{N}}\bigcup_{n=m}^{\infty}\{|X_n-X| \leq \frac{1}{n}\} &\subset  \bigcup_{k\in\mathbb{N}>0}\bigcap_{m\in\mathbb{N}}\bigcup_{n=m}^{\infty}\{|X_n-X| \leq \frac{1}{k}\}\\
		\Rightarrow & = P\left(\bigcap_{m\in\mathbb{N}}\bigcup_{n=m}^{\infty}\{|X_n-X| \leq \frac{1}{n}\}\right) &= 0
	\end{split}
\end{equation}

% neuer Abschnitt?

Also
\begin{equation}
	\begin{split}
		0 &= P\left(\bigcap_{m\in\mathbb{N}}\bigcup_{n=m}^{\infty}\{|X_n-X| \leq \frac{1}{n}\}\right) \\
			&= \lim_{m\rightarrow\infty} P\left(P\left(\bigcup_{n=m}^{\infty}\{|X_n-X| \leq \frac{1}{n}\}\right)\right) \\
			&\geq \limsup_{m\rightarrow\infty} P\left(|X_n-X| \leq \frac{1}{n}\right)
	\end{split}
\end{equation}

\section{Chebyshevsches schwaches Gesetz der gro�en Zahlen}

Seien $X_n$ unabh�ngige, reelle zufallsvariablen, je mit derselben Verteilung. Es sei $E(|X_1|^2)<\infty$.
Sei $X_n=\frac{1}{n}\cdot \sum_{j=1}^{n}{X_j},\ n\in\mathbb{N}, \mu = E(X_1)$. Dann gilt % X quer?
\begin{equation}
	X_n \stackrel{P}{\rightarrow}\mu % x quer
\end{equation}

\subsection{Beweis}
Mit der Chebyshevschen Ungleichung und $\mu=E(X_n)$. % x quer
\begin{equation}
	P(|X_n-\mu|) \leq \frac{1}{\epsilon^2}Var(X_n)=\frac{1}{\epsilon^2\cdot n^2} \sum_{j=1}^n{Var(X_j)} = \frac{1}{\epsilon\cdot n}Var(X_1) \stackrel{\rightarrow}{(n,\ldots,\infty)} 0
\end{equation}

\section{Starkes Gesetz der gro�en Zahlen}
von Kolmogorov, 1933.

Seien $X_n$ unabh�ngige reelle Zufallsvariablen, ke mit derselben Verteilung mit endlichen Erwartungswert $\mu$. Dann gilt
\[ X_n\rightarrow\mu, P\text{-f.s.} \] % x quer

\subsection{Anwendung}

Berechnung von Integralen mittels Monte-Carlo-Simulation

\[g:R^k\rightarrow \mathbb{R}, Q \leq g \leq 1\]

Es sei \[ \int_{-\infty}^{+\infty}\cdots\int_{-\infty}^{+\infty}{g(x_1,\ldots,x_n)\cdot dx_1\cdots dx_k} = \mu < \infty \]

Es ser $f > 0 $ die Dichte eines $k$-dim. Zufallsvektors $X$.
Dann ist 
\[ \mu = \int_{-\infty}^{+\infty}\cdots\int_{-\infty}^{+\infty}{\underbrace{\frac{g(x_1,\ldots,x_k}{f(x_1,\ldots,x_k)}}_{= h(x_1,\ldots,x_k)}\cdot f(x_1,\ldots,x_k)\cdot dx_1\cdots dx_k} = E(h(X)) \]

% neuer abschnitt

Seien $X_1,X_2,\ldots$ unabh., je mit derselben Vtl. wie $X$.

$Y_n=h(X_n), n \in \mathbb{N}$ die $X_n$ sind unabh. je mit derselben Vtl. mit $\mu = E(Y_1)$.

\[ \frac{1}{1}\cdot Y_n = \frac{1}{n}\cdot\sum_{j=1}^n{h(X_j}\rightarrow\mu, P-f.s. \] % y quer

Verschaffe Beobachtung $x1,\ldots,x_n$ von $X_1,\ldots,X_n$ und fasse 

$\frac{1}{n}\cdot\sum_{j=1}^{n}h(x_j) $ als Approximation f�r $\mu =\int\int{g(x_1,\ldots,x_k)dx_1\cdots dx_k}$ % int von -inf +ingf  
auf