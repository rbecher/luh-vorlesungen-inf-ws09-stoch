\part{Einf�hrung / Organisatorisches}
\label{sec:Einf�hrungOrganisatorisches}

\par Die Klausur wird am \textbf{01.02.2010 von 9:15 - 10:45} stattfinden, es wird w�chentliche Haus- und Stunden�bungen geben.\\
Die abgegebenen Haus�bungen werden korrigiert und in den �bungsgruppen zur�ckgegeben werden, es gibt allerdings keine Bonuspunkteregelung.
�bungsgruppen wird es folgende geben:
\begin{itemize}
	\item Mo, 12:15 - 13:00 Uhr (F309)
	\item Mo, 14:15 - 15:00 Uhr (G123) [max. 20 Teilnehmer]
	\item Di, 10:15 - 11:00 Uhr (F428)
	\item Di, 12:15 - 13:00 (F442)
	\item Mi, 12:15 - 13:00 (B305)
\end{itemize}

\newpage

\part{Wahrscheinlichkeitstheorie}
\label{sec:Wahrscheinlichkeitstheorie}

"`Was ist das richtige Modell, um ein Experiment zu beschreiben?"' ist die wichtigste Frage, die die mathematische Statistik zu beantworten hat, und diese Modelle an die Wirklichkeit anzupassen, ist die haupts�chliche Aufgabe der Stochastik. Der letzte Teil wird vor allem in der Vorlesung \textbf{Stochastik B} behandelt.

\section{Wahrscheinlichkeitsr�ume}
\label{sec:Wahrscheinlichkeitsr�ume}

Zufallsexperimente werden beschrieben durch
\begin{itemize}
	\item die Menge der m�glichen Ergebnisse $\Omega$\\
	Im \textbf{W�rfelwurf} sei beispielsweise $\Omega := \{1,2,3,4,5,6\}$
	\item die Menge $\mathfrak{A}$ $\subset \mathfrak{P}(\Omega):=\{A; A\subset\Omega\}$ als Menge der interessierenden Ergebnisse.\\
	$\mathfrak{A}$ ist eine so genannte \href{http://de.wikipedia.org/wiki/Sigma-Algebra}{$\sigma$-Algebra} auf $\Omega$, d.h. 
	\begin{enumerate}
	\item $\Omega \in \mathfrak{A}$.\\(Die Grundmenge $\Omega$ ist in $\mathfrak{A}$ enthalten)
	\item Mit $A \in \mathfrak{A} \Leftrightarrow A^c \in \mathfrak{A}$.\\ (Wenn $\mathfrak{A}$ eine Teilmenge $A\in\Omega$ als m�gliches Ergebnis enth�lt, dann auch deren Komplement $A^c=\Omega \setminus A$)
	\item Mit $A_1,A_2,\dots\in \mathfrak{A}$ ist auch $\bigcup_{n=1}^{\infty}A_n\in \mathfrak{A}$.\\
	(Wenn f�r jede nat�rliche Zahl $n$ die Menge $A_n \in \mathfrak{A}$ ist, so ist auch die abz�hlbare Vereinigung aller $A_n\in\mathfrak{A}$)
\end{enumerate}
	Im \textbf{W�rfelwurf} seien beispielsweise g�ltige Ereignisse:
	\begin{itemize}
	\item $A := \{2,4,6\}$ das Ergeignis "`Es f�llt eine gerade Zahl"', und
	\item $A := \{5,6\}$ das Ergeignis "`Es f�llt eine Zahl gr�\ss er gleich 5"'.
\end{itemize}
	\item durch ein \href{http://de.wikipedia.org/wiki/Wahrscheinlichkeitsmass}{Wahrscheinlichkeitsma\ss}\ $P$ (lat: "`Probabilitas"'=Glaubw�rdigkeit) mit \linebreak[4] $P:\mathfrak{A} \rightarrow [0,1]$, also eine Mengenfunktion auf $\mathfrak{A}$ mit den Eigenschaften
	\begin{enumerate}
	\item $P(\Omega)=1$ (die so genannte "`Normiertheit"').
	\item F�r jede Folge von \textbf{paarweise disjunkten} $A_1,A_2,\dots\in \mathfrak{A}$ gilt:
	\begin{equation}
		P\left(\bigcup_{n=1}^{\infty}A_n\right)=\sum_{n=1}^{\infty}{P(A_n)}
	\end{equation}
	\[\sigma\text{-Additivit�t von }P\]
	\noindent F�r $A\in \mathfrak{A}$ ist $P(A)$ die Wahrscheinlichkeitsfunktion (f�r das Eintreten) von A
\end{enumerate}
\end{itemize}

\par\noindent Das Tripel $(\Omega,\mathfrak{A},P)$ hei\ss t \href{http://de.wikipedia.org/wiki/Wahrscheinlichkeitsraum}{\textbf{Wahrscheinlichkeitsraum}}.\\

\par\noindent $\omega\in\Omega$ nennen wir das Ergebnis des Zufallsexperimentes. Das Ereignis $A\in \mathfrak{A}$ tritt genau dann ein, wenn $\omega\in A$.
\begin{equation}
\label{ }
\omega\in\bigcup_{n=1}^{\infty}A_n \iff \exists n\in \mathbb{N} \wedge \omega\in A_n
\end{equation}
Das durch $\bigcup_{n=1}^{\infty}A_n$ beschriebene Ereignis tritt genau dann ein, wenn mindestens eines der Ereignisse $A_n,\ n\in N$ eintritt.

\noindent Auf die Frage, ob auch $\emptyset$ eine g�ltige Menge ist, sei gesagt, dass \[\emptyset =\Omega^c \in \mathfrak{A}\] das "`unm�gliche Ereignis"' genannt wird (Die Antwort ist also ``ja'').\\

\noindent Seien $A_1,A_2,\ldots \in \mathfrak{A}$. Dann 
\begin{equation}
\bigcap_{n=1}^{\infty}{A_n}=\left(\bigcup_{n=1}^{\infty}A_n^c\right)^c\in \mathfrak{A}
\end{equation}
Anmerkung: $\bigcap_{n=1}^{\infty}A_n$ tritt genau dann ein, wenn alle $A_n,n\in N$ eintreten.\\

\par\noindent Wenn nun $A,B \in \mathfrak{A}$, dann gilt
\begin{itemize}
	\item \[A\cup B = A\cup B \cup \emptyset \cup \emptyset \cup \dots \in \mathfrak{A}\]
	(Die Vereinigung von $A,B$ ist auch ein interessierendes Ereignis und steht f\"ur ``das Ereignis $A$ oder $B$ treten ein'')
	\item \[A\cap B = A\cap B \cap \Omega \cap \Omega \cap \dots \in \mathfrak{A}\]
	(Die Schnittmenge von $A,B$ ist auch ein interessierendes Ereignis und steht f\"ur ``das Ereignis $A$ \textbf{und} $B$ treten ein'')
	\item \[A \Delta B = (A\cap B^c) \cup (B\cap A^c)\]
	(Die ``symmetrische Differenz'' beschreibt das Ereignis ``Entweder A oder B'')
	\item $A \subseteq B$ hei{\ss}t: ``Das Ereignis A hat das Ereignis B zur Folge''.\\
	(Kurz erkl\"art: Wenn $A$ eintritt, folgt $\omega \in A\ \wedge\ A \subseteq B \Rightarrow \omega \in B$)
\end{itemize}

\subsection{Eigenschaften von Wahrscheinlichkeits-Ma{\ss}en}

\begin{itemize}
	\item Aus $P(\emptyset)=P(\emptyset\cup\emptyset\cup\dots)=\sum_{n=1}^{\infty}{P(\emptyset)}$ folgt $P(\emptyset)=0$.
	\item F�r $A,B\in \mathfrak{A}$ mit $A\subseteq B$ ist 
	\begin{equation}
P(B)=P(A \cup B \cap A^c)=P(A \cup B \cap A^c \cup \emptyset \cup \dots)=P(A)+P(B \cap A^c) \geq P(A)
\end{equation}
	Also gelten folgende Eigenschaften / Gesetzm\"a{\ss}igkeiten
	\begin{itemize}
	\item $P(A)\leq P(B)$ (Isotonie)
	\item $P(B \cap A^c)=P(B)-P(A)$ (Subtraktivit�t)
\end{itemize}
\end{itemize}

\subsection{Beispiel}

$\emptyset \neq \Omega$ sei endlich, $|\Omega|$ Anzahl der Elemente von $\Omega$ und $\mathfrak{A}=\mathfrak{P}(\Omega)$.\\
Sei $P$ definiert durch 
\begin{equation}
P(A)=\frac{|A|}{|\Omega|},\ A \subset \Omega
\end{equation}
$P$ hei\ss t "`diskretes Laplace'sches W-Ma\ss "` auf $\Omega$.\\
Es ist 
\begin{equation}
P(\{\omega\})=\frac{1}{|\Omega|},\ \omega\in\Omega
\end{equation}
also folgert man: "`Alle m�glichen Ergebnisse sind \textbf{gleich wahrscheinlich}."'

\newpage

\section{Grundformeln der Kombinatorik}
\label{sec:Grundformeln der Kombinatorik}

$n,r\in N$, $M_n$ ist eine $n$-elementige Menge, o.B.d.A. $M_n=\{1,\ldots,n\}$.
\begin{itemize}
	\item 
	\begin{equation}
P_n^{(r)}=\{(x_1,\dots,x_r);\ x_i \in M_n, 1 \leq i \leq r\}
\end{equation}
	das sind die $r$-\href{http://de.wikipedia.org/wiki/Permutation}{Permutationen} aus $M_n$ mit Wiederholzung.
	\item 
	\begin{equation}
	P_n^{(r)}=\{(x_1,\dots,x_r);\ x_i \in M_n, 1 \leq i \leq r, \text{paarweise\ verschieden}\}\ (r \leq n)
\end{equation}
	das sind die $r$-Permutationen aus $M_n$ ohne Wiederholung.
	\item 
	\begin{equation}K_n^{(r)}=\{(x_1,\dots,x_r);\ x_i \in M_n, 1 \leq i \leq r,\ x_1 < x_2 < \dots < x_n\}\ (r \leq n)
\end{equation}
	das sind die $r$-Kombinationen aus $M_n$ ohne Wiederholung.
	\item 
	\begin{equation}K_n^{(r)}=\{(x_1,\dots,x_r);\ x_i \in M_n, 1 \leq i \leq r,\ x_1 \leq x_2 \leq \dots \leq x_n\}\ (r \leq n)
\end{equation}
	das sind die $r$-Kombinationen aus $M_n$ mit Wiederholung.
\end{itemize}

\subsection{Zur Erkl\"arung}
\begin{enumerate}
	\item Tupel in Permutationen sind nicht angeordnet, hier ist die Reihenfolge wichtig!
  \item Tupel in Kombinationen sind o.B.d.A. angeordnet, wir k\"onnen sie also ``vergleichen''.
  \item Bei Varianten \textbf{mit Wiederholung}, d\"urfen also zwei Elemente ``nebeneinander'' auch gleich sein, bei Varianten \textbf{ohne Wiederholung} ist das nicht der Fall, hier m\"ussen die Elemente also echt unterschiedlich sein.
  \item Bei einer $r$-Permutation/Kombination haben wir eine $r$-elementige Teilmenge (die also kleiner oder gleich gro{\ss} der ``gro{\ss}en'' Menge $M_n$ ist).
  \item ``mit Wiederholung'' nennt man auch ``mit Zur\"ucklegen''
\end{enumerate}

\subsection{Anschauliche Beispiele}
\begin{itemize}
	\item Bei Kombinationen interessiert uns ausschlie�lich, welche Elemente wir aus $M_n$ bekommen, ein gutes Beispiel f�r eine Kombination
	\begin{description}
	\item[ohne Wiederholung] ist also das �bliche Lotto-Spiel ``6 aus 49''. Hier ist es uns egal, ob die Reihenfolge $(1,2,3,4,5,6)$ ist oder $(6,5,4,3,2,1)$.
	\item[mit Wiederholung] ist die Frage, ob wir mindestens zwei Mal in f�nf Versuchen eine Zahl $\geq 5$ w�rfeln
\end{description}
	\item Bei Permutationen interessiert uns neben der Auswahl auch die Anordnung (man sagt: die Elemente sind ``unterscheidbar''). Eine Permutation
	\begin{description}
	\item[ohne Wiederholung] ist z.B. ein Kinobesuch. Ob Tim zwischen Tina und Claudia sitzt, ist f�r ihn nicht gleich bedeutend, wie zwischen Klaus und J�rgen zu sitzen.
	\item[mit Wiederholung] ist die Frage, ob ich beim Roulette einen aufsteigenden Run habe, also erst eine $0$ f�llt, dann eine $1$, dann eine $2$, usw. 
\end{description}
\end{itemize}

\subsection{Rechnen mit Permutationen und Kombinationen}
Es gilt
\begin{itemize}
	\item $|P_n^r|=n^r$
	\item $|P_n^{(r)}|=n(n-1)\ldots(n-r+1)$, im Fall $n=r$: $|P_n^{(r)}|=r!$
	\item $r!|K_n^{(r)}|=|P_n^{(r)}|\ \Rightarrow\ |K_n^{(r)}|=\frac{n(n-1)\dots(n-r+1)}{r!}=(\stackrel{n}{r})$\\
	Dann gilt auch $|\{A; A\subset M_n, |A|=r\}|^r=\stackrel{n}{r}$. Gilt auch noch f�r $r=0$\\
	Dann folgt: $|\mathfrak{P}(M_n)|=\sum_{r=0}^{r}{\stackrel{n}{r}1^r1^{n-r}}=(1+1)^n=2^n$
	\item $K_n^r \rightarrow (x_1,\dots,x_r) \underbrace{\longrightarrow}_{ist\ bijektiv} (x_1,x_1+1,\dots,x_r+r-1)\in K_{n+r-1}^{(r)}$\\
	$\Rightarrow |K_n^r| = |K_{n+r-1}^{(r)}| = \stackrel{n+r-1}{r}$\\
	Identifiziere $(x_1,\dots,x_r) \in K_n^r$ mit dem Besetzungszahlvektor $(k_1,\dots,k_n)$,
	wobei $k_i=|\{j\in \{ 1,\dots,r \}; x_j=i\}|,\ 1 \leq i \leq n$. Dann ist $K-1+\dots+k_n=r$.\\
	Dann ist $|\{(K_1,\dots,k_n)\in N_0^n; k_1+\dots+k_n=r\}|=\stackrel{n+r-1}{r}$\\
	Folgerung: $|\{(K_1,\dots,k_n)\in N^n; k_1+\dots+k_n=r\}|=\stackrel{n+(r-n)-1}{r-n}=\stackrel{r-1}{r-n}=\stackrel{r-1}{n-1}, r \geq n$\\
\end{itemize}