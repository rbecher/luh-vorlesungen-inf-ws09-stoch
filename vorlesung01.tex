\part{Einf�hrung}
\label{sec:Einf�hrung}

\section{Noch leer}
\label{sec:Mengen}

\subsection{Die Formel von Sylvester-Poincar�}
\label{sec:Siebformel}

\par Und hier kommt schon die schwierigste Formel in der ganzen Vorlesung:
\begin{displaymath}
\mathsf{P}\left(\bigcup_{i=1}^nA_i\right) = \sum_{k=1}^n(-1)^{k+1}\!\!\sum_{I\subseteq\{1,\dots,n\},\atop |I|=k}\!\!\!\!\mathsf{P}\left(\bigcap_{i\in I}A_i\right)
\end{displaymath}
\par Blubb

$\sum_i^ni=\frac{n(n+1)}{2}$

\subsection{Spielereien mit fancybox}
\label{sec:fancybox}

\begin{displaymath}
	x + y = \fbox{$\Omega$}
\end{displaymath}

\begin{equation}
\fbox{$\displaystyle
\int_{\Omega_0} \zeta(\omega) d\omega
\geq \bar{r}$}
\end{equation}

\newlength{\mylength}
\[
\setlength{\fboxsep}{15pt}
\setlength{\mylength}{\linewidth}
\addtolength{\mylength}{-2\fboxsep}
\addtolength{\mylength}{-2\fboxrule}
\fbox{%
\parbox{\mylength}{
\setlength{\abovedisplayskip}{0pt}
\setlength{\belowdisplayskip}{0pt}
\begin{equation}
x + y = z
\end{equation}}}
\]

\newenvironment{FramedEqn}%
{\setlength{\fboxsep}{15pt}
\setlength{\mylength}{\linewidth}%
\addtolength{\mylength}{-2\fboxsep}%
\addtolength{\mylength}{-2\fboxrule}%
\Sbox
\minipage{\mylength}%
\setlength{\abovedisplayskip}{0pt}%
\setlength{\belowdisplayskip}{0pt}%
\equation}%
{\endequation\endminipage\endSbox
\[\fbox{\TheSbox}\]}

\begin{FramedEqn}
\Rightarrow P\sim\xi(P_\gamma)- \frac{1}{3}
\end{FramedEqn}

\fbox{%
\begin{Beqnarray*}
x & = & y\\
y & > & x \\
\int_4^5 f(x)dx & = & \sum_{i\in F} x_i
\end{Beqnarray*}}