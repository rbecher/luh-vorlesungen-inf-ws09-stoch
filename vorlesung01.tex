\part{Einf�hrung / Organisatorisches}
\label{sec:Einf�hrung / Organisatorisches}

\par Die Klausur wird am \textbf{01.02.2010 von 9:15 - 10:45} stattfinden, es wird w�chentliche Haus- und Stunden�bungen geben.\\
Die abgegebenen Haus�bungen werden korrigiert und in den �bungsgruppen zur�ckgegeben werden. 
�bungsgruppen wird es folgende geben:
\begin{itemize}
	\item Mo, 12:15 - 13:00 Uhr (F309)
	\item Mo, 14:15 - 15:00 Uhr (G123) [max. 20 Teilnehmer]
	\item Di, 10:15 - 11:00 Uhr (F428)
	\item Di, 12:15 - 13:00 (F442)
	\item Mi, 12:15 - 13:00 (B305)
\end{itemize}

\part{Wahrscheinlichkeitstheorie}
\label{sec:Wahrscheinlichkeitstheorie}

"`Was ist das richtige Modell, um ein Experiment zu beschreiben?"' ist die wichtigste Frage, die die mathematische Statistik zu beantworten hat, und diese Modelle an die Wirklichkeit anzupassen, ist die haupts�chliche Aufgabe der Stochastik. Der letzte Teil wird vor allm in der Vorlesung \textbf{Stochastik B} behandelt.

\section{Wahrscheinlichkeitsr�ume}
\label{sec:Wahrscheinlichkeitsr�ume}

Zufallsexperimente werden beschrieben durch
\begin{itemize}
	\item die Menge der m�glichen Ergebnisse $\Omega$\\
	Im \textbf{W�rfelwurf} sei beispielsweise $\Omega := \{1,2,3,4,5,6\}$
	\item die Menge A $\subset \mathfrak{P}(\Omega):=\{A; A\subset\Omega\}$ als Menge der interessienden Ergebnisse\\ % S�tterlin-A undR
	$\mathfrak{A}$ ist eine so genannte $\sigma$-Algebra auf $\Omega$, d.h. 
	\begin{enumerate}
	\item $\Omega \in \mathfrak{A}$
	\item Mit $A \in \mathfrak{A} \Leftrightarrow A^c \in \mathfrak{A}$
	\item Mit $A_1,A_2,\dots\in \mathfrak{A}$ ist auch $\bigcup_{n=1}^{\infty}A_n\in \mathfrak{A}$
\end{enumerate}
	Im \textbf{W�rfelwurf} sei beispielsweise $A := \{2,4,6\}$ das Ergeignis "`Es f�llt eine gerade Zahl"' und 
	\item durch ein Wahrscheinlichkeitsma\ss\ $P$ (lat: "`Probabilitas"'=Glaubw�rdigkeit) mit \nolinebreak{$P:\mathfrak{A} \rightarrow [0,1]$}, das ist eine Mengenfunktion auf $\mathfrak{A}$ mit den Eigenschaften
	\begin{enumerate}
	\item $P(\Omega)=1$
	\item F�r jede Folge von \textbf{paarweise disjunkten} $A_1,A_2,\dots\in \mathfrak{A}$ gilt:\\
	$P\left(\bigcup_{n=1}^{\infty}A_n\right)=\sum_{n=1}^{\infty}{P(A_n)}$\ \ \ ($\sigma$-Additivit�t von $P$)\\
	F�r $A\in \mathfrak{A}$ ist $P(A)$ die Wahrscheinlichkeitsfkt. (f�r das Eintreten) von A
\end{enumerate}
\end{itemize}

Das Tripe $(\Omega,\mathfrak{A},P)$ hei\ss t \textbf{Wahrscheinlichkeitsraum}. 

\par\noindent $\omega\in\Omega$ nennen wir das Ergebnis des Zufallsexperimentes Das Ereignis $A\in \mathfrak{A}$ tritt genau dann ein, wenn $\omega\in A$.\\
$\omega\in\bigcup_{n=1}^{\infty}A_n \Leftrightarrow$ Es gibt ein $n\in N$ mit $\omega\in A_n$\\
Das durch $\bigcup_{n=1}^{\infty}A_n$ beschriebene Ereignis tritt genau dann ein, wenn mindestens eines der Ereignisse $A_n, n\in N$ eintritt.

Auf die Frage, ob auch $\emptyset$ eine g�ltige Menge ist, sei gesagt, dass $\emptyset =\Omega^c \in \mathfrak{A}$ das "`unm�gliche Ereignis"' genannt wird.

Seien $A_1,A_2,\dots \in \mathfrak{A}$. Dann $\bigcap{n=1}^{\infty}A_n=\left(\bigcup_{n=1}^{\infty}A_n^c\right)^c\in \mathfrak{A}$.\\
Anmerkung: $\bigcap_{n=1}^{\infty}A_n$ tritt genau dann ein, wenn alle $A_n,n\in N$ eintreten.

Wenn nun $A,B \in \mathfrak{A}$, dann
\begin{itemize}
	\item $A\cup B = A\cup B \cup \emptyset \cup \emptyset \cup \dots \in \mathfrak{A}$
	\item $A\cap B = A\cap B \cap \Omega \cap \Omega \cap \dots \in \mathfrak{A}$
	\item $A \Delta B = (A\cap B^c) \cup (B\cap A^c)$ (Die symmetrische Differenz) beschreibt das Ereignis "`Entweder A oder B"'
	\item $A \subset B$ hei\ss t: "`Das Ereignis A hat das Ereignis B zur Folge"' % ???
\end{itemize}

\subsection{Eigenschaften von W-Ma\ss en}

\begin{itemize}
	\item Aus $P(\emptyset)=P(\emptyset\cup\emptyset\cup\dots)=\sum{P_{n=1}^{\infty}(\emptyset)}$ folgt $P(\emptyset)=0$.\\
	\item F�r $A,B\in \mathfrak{A}$ mit $A\subset B$ ist $P(B)=P(A \cup B \cap A^c)=P(A \cup B \cap A^c \cup \emptyset \cup \dots)=P(A)+P(B \cap A^c) \geq P(A)$\\
	Also 
	\begin{itemize}
	\item $P(A)\leq P(B)$ (Isotonie)
	\item $P(B \cap A^c)=P(B)-P(A)$ (Subtraktivit�t)
\end{itemize}
\end{itemize}

\subsection{Beispiel}

$\emptyset \neq \Omega$ sei endlich, $|\Omega|$ Anzahl der Elemente von $\Omega$, $\mathfrak{A}=\mathfrak{P}(\Omega)$. Sei $P$ definiert durch\\
$P(A)=\frac{|A|}{|\Omega|},\ A \subset \Omega$\\
$P$ hei\ss t "`diskretes Lapacesches W-Ma\ss "` auf $\Omega$.\\
Es ist $P(\{\gamma\})=\frac{1}{|\Omega|}, \omega\in\Omega$, also folgert man:\\
"`Alle m�glichen Ergebnisse sind \textbf{gleich wahrscheinlich}."'

\section{Grundformeln der Kombinatorik}
\label{sec:Grundformeln der Kombinatorik}

$n,r\in N$, $M_n$ ist eine $n$-elementige Menge, o.B.d.A. $M_n=\{1,\dots,n\}$.
\begin{itemize}
	\item $P_n^r=\{(x_1,\dots,x_r);\ x_i \in M_n, 1 \leq i \leq r\}$, das sind die $r$-Permutationen aus $M_n$ mit Wiederholzung.
	\item $P_n^{(r)}=\{(x_1,\dots,x_r);\ x_i \in M_n, 1 \leq i \leq r,\ paarweise\ verschieden\}$ (im Falle $r \leq n$), das sind die $r$-Permutationen aus $M_n$ ohne Wiederholung.
	\item $K_n^{(r)}=\{(x_1,\dots,x_r);\ x_i \in M_n, 1 \leq i \leq r,\ x_1 < x_2 < \dots < x_n\}$ (im Falle $r \leq n$), das sind die $r$-Kombinationen aus $M_n$ ohne Wiederholung.
	\item $K_n^{(r)}=\{(x_1,\dots,x_r);\ x_i \in M_n, 1 \leq i \leq r,\ x_1 \leq x_2 \leq \dots \leq x_n\}$ (im Falle $r \leq n$), das sind die $r$-Kombinationen aus $M_n$ mit Wiederholung.
\end{itemize}
Es gilt
\begin{itemize}
	\item $|P_n^r|=n^r$
	\item $|P_n^{(r)}|=n(n-1)\cdots(n-r+1)$, im Fall $n=r$: $|P_n^{(r)}|=r!$
	\item $r!|K_n^{(r)}|=|P_n^{(r)}|\ \Rightarrow\ |K_n^{(r)}|=\frac{n(n-1)\dots(n-r+1)}{r!}=(\stackrel{n}{r})$\\
	Dann gilt auch $|\{A; A\subset M_n, |A|=r\}|^r=\stackrel{n}{r}$. Gilt auch noch f�r $r=0$\\
	Dann folgt: $|\mathfrak{P}(M_n)|=\sum_{r=0}^{r}{\stackrel{n}{r}1^r1^{n-r}}=(1+1)^n=2^n$
	\item $K_n^r \rightarrow (x_1,\dots,x_r) \underbrace{\longrightarrow}_{ist bijektiv} (x_1,x_1+1,\dots,x_r+r-1)\in K_{n+r-1}^{(r)}$\\
	$\Rightarrow |K_n^r| = |K_{n+r-1}^{(r)}| = \stackrel{n+r-1}{r}$\\
	Identifiziere $(x_1,\dots,x_r) \in K_n^r$ mit dem Besetzungszahlvektor $(k_1,\dots,k_n)$,
	wobei $k_i=|\{j\in \{ 1,\dots,r \}; x_j=i\}|,\ 1 \leq i \leq n$. Dann ist $K-1+\dots+k_n=r$.\\
	Dann ist $|\{(K_1,\dots,k_n)\in N_0^n; k_1+\dots+k_n=r\}|=\stackrel{n+r-1}{r}$\\
	Folgerung: $|\{(K_1,\dots,k_n)\in N^n; k_1+\dots+k_n=r\}|=\stackrel{n+(r-n)-1}{r-n}=\stackrel{r-1}{r-n}=\stackrel{r-1}{n-1}, r \geq n$\\
\end{itemize}