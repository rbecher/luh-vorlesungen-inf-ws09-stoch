\par\noindent Aus der Menge der Permutationen betrachten wir eine spezifische Teilmenge.\\
Sei $k_1,\ldots,k_n \in \mathbb{N}_0$ mit $k_1+\ldots+k_n=r$. Einen so genannter ``Besetzungszahlvektor'' dr�ckt man beispielsweise so aus:
\begin{equation*}
	\begin{split}
	a = b
	\end{split}
\label{eq:}
\end{equation*}
\begin{equation*}
	\begin{split}
	& \{(x_1,\ldots,x_n)\in\mathbb{P}_n^r;\ |\{j; j\in\{1,\ldots,r\}, x_j=i\}|=k_i, 1\leq i\leq n\}|
%	=	& \{(x_1,\ldots,x_n)\in\mathbb{P}_n^r;\ \text{Genau $k_i$ der $r$ Komponenten $(x_1,\ldots,x_r)$ sind gleich $i$}, 1\leq i \leq n\}\\
	\end{split}
\end{equation*}

Die L�nge oder M�chtigkeit dieser Menge ist wie folgt:
\begin{equation*}
	\begin{split}
	 =	& \binom{r}{k_1}\binom{r-k_1}{k_2}\binom{r-k_1-k_2}{k_3}\cdots\binom{r-k1-\ldots-k_{n-1}}{k_n}\\
	 =	&	\frac{r!}{k_1!(r-k_1)!}\cdot \frac{(r-k_1)!}{k_2!(r-k_1-k_2)!}\cdots = \frac{r!}{k_1}!\cdots k_n! % (r-k_1)! durchstreichen ... strike?
	\end{split}
\end{equation*}

% Binomischer Lehrsatz
% => Multinomialer Lehrsatz durch Verallgemeinerung

\begin{equation}
n^r = \sum_{(k_1,\ldots,k_n)\in\mathbb{N}_0^n\\k_1+\ldots+k_n=r}{\frac{r!}{k_1!\cdots k_n!}} % Anzahl der Summanden = n+r-1 �ber r
\end{equation}

\subsection{Beispiel}
O.B.d.A. sind $r$ Personen im Raum, $n=365$ seien die Anzahle der Tage im Jahr und $r\leq n$.\\
Gesucht: Wahrscheinlickeit, dass mindestens $2$ der $r$ Personen am gleichen Tag Geburtstag haben.
\begin{equation}
\Omega = \{ (x_1,\ldots,x_r); x_i \  \}
\end{equation}
$\S�tt-A=S�tt-P(\Omega)$\\
$P(A)=\frac{|A|}{|\Omega|}, A\subseteq\Omega$

"`Alle m�glichen r-Tupel von Geburtstagen sind gleich wahrscheinlich"'
\begin{equation}
A= \{ (x_1,\ldots,x_r)\in\Omega; i,j \in \{1,\ldots,r\}, i\neq j, x_i=x_j \}
\end{equation}
\begin{equation}
	P(A) = 1-P(A^c)
\end{equation}

\begin{equation}
	A^c = \{ (x_1,\ldots,x_r)\in\Omega; x_1,\ldots,x_r \text{paarweise verschieden} \}
\end{equation}

\begin{equation}
|A^c|=n(n-1)\cdots(n-r+1)
P(A)=1- \frac{n(n-1)\cdots(n-r+1)}{n^r}
\end{equation}

ab $r>23$ ist $P(A)>1/2$\\
	 $r=30$ ist $P(A)=0,706$\\
	 $r=60$ ist $P(A)=0,994$
	 
Seien $A_1,A_n$ Ereignisse in W-Raum $(\Omega,S�tt-A,P)$\\
$P(A_1\cup A_2)$\\
$= P(A_1 \cup (A_2 \cap (A_1 \cap A_2)^c))$\\
$= P(A_1) + P(A_2 \cap (A_1 \cap A_2)^c)$\\
$= P(A_1) + P(A_2) - P(A_1 \cap A_2)$

\subsection{Die Siebformel}
Seien $A_1,\ldots,A_n$ Ereignisse\\
\begin{equation}
	P(\bigcup_{k=1}^n{A_k}) ) \sum_{k-1}^n{(-1)^{k-1}\sum_{1\leq j_1<\ldots<j_k\leq n}{P(A_{j_1}\cap \ldots \cap A_{j_k})}}
\end{equation}

\subsection{Beispiel}
$\Omega = \{ \pi; \pi:\{1,\ldots,n\} \rightarrow \{1,\ldots,n\} \text{bijektiv},\ |\Omega|=n!$\\
$\mathfrak{A}=\mathfrak{P}(\Omega)$\\
$P$ das diskrete Laplace'sche W-Ma\ss  auf $\Omega$\\
$A_i= \{ \pi \in \Omega; \pi(i)=i\},\ 1\leq i \leq n \}$\\
$\bigcup_{i=1}^n{A_i} = \{ \pi\in\Omega; \text{Es gibt ein $i\in\{1,\ldots,n\}$ mit $\pi(i)=i$}\}$\\
$P(\bigcup_{i=1}^n{A_i})=\sum_{k-1}^n{(-1)^{k-1}\sum_{1\leq j_1<\ldots<j_k\leq n}{\frac{|A_{j_1}\cap\ldots\cap A_{j_k}|}{|\Omega|}}}$\\
$= \sum_{k=1}^n{(-1)^{k-1}\sum_{...}{\frac{(n-k)!}{n!}}}=\sum_{k=1}^n{(-1)^{k-1}\binom{n}{k}\cdot \frac{(n-k)!}{n!}}=\sum_{k=1}^n{(-1)^{k-1}\frac{1}{k!}}$

$\bigcap_{i=1}^n{A_i^c} = \left(\bigcup_{i=1}^nA_i\right)^c=\{\pi \in \Omega; \forall i \in \{1,\ldots,n\}\text{ist} \pi(i)\neq i\}$\\
$P(\bigcap_{i=1}^n{A_i^c})=1-\sum_{k=1}^n{(-1)^{k-1} \frac{1}{k!}}=\sum_{k=0}^n{(-1)^{k} \frac{1}{k!}} \stackrel{\rightarrow}{(n\rightarrow\infty)}\sum_{k=0}^{\infty}{(-1)^k \frac{1}{k!}}=e^{-1} \approx 0.37$

\subsection{Weitere Beispiele f�r W-Ma�e}

$\emptyset\neq\Omega$, $\mathfrak{A}$ $\sigma$-Algebra mit $\{\omega\}\in \mathfrak{A}$ f�r jedes $\omega\in\Omega$.\\
Sei $P$ W-Ma� auf $\Omega$ mit der Eigenschaft, dass eine abz�hlbare Menge $\Omega_o\subseteq\Omega$ existiert mit:\\
$P(\Omega_0)=1$\\
Dann $P(\Omega_0^c)=1-P(\Omega_0^c)=0$
F�r $A\in \mathfrak{A}$ gilt:\\
$P(A)=P((A\cap\Omega_0)+(A\cap\Omega_0^c))$\\
$= P(A\cap\Omega_0) + P(A\cap\Omega_0^c)$\\ % last is disjunkt
$= P(\sum_{\omega\in A\cap\Omega_0}{\{\omega\}})=\sum_{\omega\in A\cap\Omega_o}{P(\{\omega\})}$\\
$=\sum_{\omega\in\Omega}{P(\{\omega\})\delta_\omega(A)}$

F�r $\omega\in\Omega$ hei�t $\delta_\omega:\mathfrak{A} \rightarrow [0,1]$, definiert durch\\
$\delta_\omega(A)=1, \omega\in A\ oder\ 0, \omega\not\in A$\\ %cases
das Einpunktma� oder Dirac-Ma� in $\omega$.\\

Es gilt also $P=\sum_{\omega\in\Omega_0}{P(\{\omega\})}\delta_\omega$.\\
Solche W-Ma�e hei�en \textbf{diskrete W-Ma�e}.

Betrachte den Fall $\Omega=\mathbb{R}$\\
$Suett-B=\mathfrak{A}=\bigcap_{\mathfrak{A}^* \sigma-Algebra auf \mathbb{R}\\\mathfrak{A}^*\supseteq\{(-\infty,x]; x\in\mathbb{R}\}}\mathfrak{A}^*$ ist die Borelsche $\sigma$-Algebra auf $\mathbb{R}$

Wichtig: $Suett-B\neq \mathfrak{P}(\mathbb{R})$

$Suett-B$ enth�lt alle offenen, abgeschlossenen und ohne Intervalle Teilmengen von R.

Sei $F:R\rightarrow R$ Funktion mit folgenden Eigenschaften:
\begin{enumerate}
	\item $F$ ist monoton wachsend
	\item $F$ ist rechtsseitig stetig
	\item $\lim_{x\rightarrow-\infty}F(x)=0; \lim_{x\rightarrow+\infty}F(x)=1$
\end{enumerate}

Es existiert genau ein W-Ma� $P$ auf $Suett-B$ (auf $\mathbb{R}$) mit der Eigenschaft, dass\\
\begin{equation}
P((-\infty,x]))=F(x),\ \forall x\in\mathbb{R}
\end{equation}

$F$ hei�t die zu $P$ geh�rige \textbf{Verteilungsfunktion}.

F�r $(a,b]$, $-\infty<a<b<+\infty$ ist
\begin{equation}
	P((a,b]) = P((-\infty,b]) - P(-\infty,a]) = F(b)-F(a)
\end{equation}

\subsection{Beispiel}

Sei $f:\mathbb{R} \rightarrow \mathbb{R}_+$ (uneigentlich) Riemann-integrierbar mit  % Bildmenge ist genau [0,\infty)
\begin{equation}
	\int_{- \infty }^{+\infty}{f(t)dt}=1
\end{equation}

Setze 
\begin{equation}
	F(x)=\int_{- \infty}^{+ \infty}{f(t)dt},\ -\infty <x<+\infty s
\end{equation}

$F$ ist Verteilungsfunktion eines W-Ma�es $P$ auf $R$.\\
$f$ hei�t Wahrscheinlichkeitsdichte (oder einfach "`Dichte"') von $P$.

\subsubsection{Spezialf�lle}

\begin{itemize}
	% a
	\item $f(t)= 0 \forall x<a,\ and\ \frac{1}{b-a} \forall a\leq x\leq b\ oder\ 0 \forall x>b$\\ % cases
				$F(x) = 0 \forall x<a,\ and\ \frac{x-a}{b-a} \forall a\leq x\leq b\ oder\ 1 \forall x>b$ % cases
	% b
	\item $f(t) = \frac{1}{\sqrt{2\pi\sigma^2}} exp(-\frac{1}{2}\frac{(t-a)^2}{\sigma^2}), -\infty<t<+\infty, (a\in R, \sigma^2>0)$\\ % das ist die glockenkurve
Dann ist $\int_{-\infty}^{+\infty}{f(t)dt=1}$\\
\begin{equation}
	\begin{split}
		F(x) 		& = \int_{-\infty}^{+\infty}{\frac{1}{\sqrt{2\pi\sigma^2}}exp(-\frac{1}{2}\frac{(t-a)^2}{\sigma^2})dt}\\
				 		& = \phi(\frac{x-a}{\sigma})\\
		\phi(x)	& = \int_{-\infty}^{+\infty}{\frac{1}{\sqrt{2\pi\sigma^2}}exp(-\frac{t^2}{2})}dt, x\in \mathbb{R}
	\end{split}
\end{equation}
\end{itemize}