\part{Erwartungswerte}

\section{Erwartungswerte von Zufallsvariablen}

Wir definieren Eerwartungswerte von Zufallsvariablen schrittweise, beginnend mit Indikatorvariablen. Es sei $X$ eine eine reelle Zufallsvariable auf dem Wahrscheinlichkeitsraum $()$.
\renewcommand{\labelenumi}{(\theenumi)}
\begin{enumerate}
	\item F�r eine Indikatorvariable$X=I_A$ mit $A \in Suett-A$ sei $E(X) = E_p(X) = P(A)$
	\item F�r eine nicht negative einfache Zufallsvariable $X=\sum_{k=1}^n{\alpha_kI_{A_k}}$ mit nicht negativen reellen Zahlen $\alpha_1,\ldots,\alpha_n$ und Ereignissen $A_1,\ldots,A_n \in Suett-A$ sei 
		\begin{equation}
			E(X) = E_p(X) = \sum{\alpha_k\cdot P(A_k)}
		\end{equation}
		Der so definierte Wert $E(X)$ h�ngt nicht von der Darstellung von $X$ ab, d.h. ist $X=$ \ldots
	\item F�r eine nicht negative Zufallsvariable $X \geq 0$ gibt es eine Folge von nicht negativen einfachen Zufallsvariablen $X_n,\ n \in \mathbb{N}$, mit $0 \leq X_1 \leq \ldots$ mit $X = \sup_{n\in\mathbb{N}}X_n$. Es sei $E(X)=E-p(X)=\sup_{n\in\mathbb{N}}E(X)$
	\item Jede reelle Zufallsvariable $X$ l�sst sich in der Form $X=X^+-X^-$ mit den nicht negativen Zufallsvariablen $X^+=\max{(X,0)}$
\end{enumerate}


\section{Eigenschaften von Erwartungswerten}

Es seien $X,Y$ reelle Zufallsvariablen auf dem Wahrscheinlichkeitsraum $()$, $\alpha,\beta$ reelle Zahlen.
\begin{enumerate}
	\item Sind $X,Y \geq 0$ und $\alpha,\beta \geq 0$, so ist \[E(\alpha\cdot X + \beta\cdot Y) = \alpha\cdot E(X) + \beta\cdot E(Y)\]
	\item Sind $X,Y$ integrierbar, so ist $\alpha\cdot X + \beta\cdot Y$ integrierbar und es ist \[E(\alpha\cdot X + \beta\cdot Y) = \alpha\cdot E(X) + \beta\cdot E(Y)\]
	\item ist $0 \leq X \leq Y$, so ist $0 \leq E(X) \leq E(Y)$.
	\item Existiert f�r die Zufallsvariable $X$ der Erwartungswert, so ist $|E(X)| \leq E(|X|)$.
	\item Sind $X,Y$ integrierbar und ist $X \leq Y$, so ist $E(X) \leq E(Y)$.
	\item $X$ ist genau dann integrierbar, wen $|X|$ integrierbar ist.
	\item Ist $|X| \leq |Y|$ und $Y$ integrierbar, so ist auch $X$ integrierbar.
	\item Sind $X,\ldots,X_n,\ n \in \mathbb{N}$ nicht negative Zufallsvariablen mit $0 \leq X_1 \leq X_2 \leq \ldots$ und $X= \sup_{n\in\mathbb{N}}{X_n}$, so ist $E(X) = \sup_{n\in\mathbb{N}}{E(X_n)}$
	\item Sind $X_n,\ n \in \mathbb{N}$ nicht negative Zufallsvariablen, so ist \[E(\liminf_{n\rightarrow\infty}{X_n}) \leq \liminf_{n\rightarrow\infty}{E(X_n)}\]
	\item Sind $X,Y,X_n,\ n\in\mathbb{N}$ reelle Zufallsvariablen mit $X=\lim_{n\rightarrow\infty}{X_n}$ und $|X_n| \leq |Y|$ f�r jedes $n \in \mathbb{N}$, und es ist $Y$ integrierbar, so sind auch $X,X_n$ integrierbar und es gilt $\lim_{n\rightarrow\infty}{E(|X_n-X|)} = 0$. Insbesondere folgt hieraus auch $\lim_{n\rightarrow\infty}{E(X)}=E(X)$.
	\item Ist $X \geq 0$, so ist $E(X)=0$ genau dann, wenn $P(X > 0) = 0$ ist.
\end{enumerate}

Aus der Eigenschaft (8) ergibt sich f�r eine Folge von nicht negativen, reellen Zufallsvariablen $X_n$ mit der Eigenschaft, dass die Reihe $\sum_{k=1}^{n}{X_k}$ gegen die reelle Zufallsvariable $X$ konvergiert, die Identit�t
\begin{equation}
	E\left( \sum_{n=1}^{\infty}{X_n} \right) = \ldots
\end{equation}
\ldots
folgt hieraus
\begin{equation}
	|X| = \sum_{n=1}{|X|\cdot I(n-1 < X \leq n)},
\end{equation}
dass
\begin{equation}
	\begin{split}
		E(|X|) &= \sum_{n=1}^{\infty}{E(|X|\cdot I(n-1 < X \leq n))} \\
					 &\leq \sum_{n=1}^{\infty}{P(n-1 < |X| \leq n)} \\
					 &= \ldots \\
					 &= \sum_{k=1}^{\infty}{\sum_{n=k}^{\infty}{P(n-1 < |X| \leq n)}} \\
					 &= \sum_{k=1}^{\infty}{P(|X| > n-1)} \\
					 &= \sum_{k=0}^{\infty}{P(|X| > n)}
	\end{split}
\end{equation}
ist. Analgo sieht man
\begin{equation}
	E() \ldots
\end{equation}

\ldots

\section{Rechnen mit Erwartungswerten}

$()$ W-Raum, $\emptyset \neq R$ Menge, $Suett-L \sigma$-Algebra auf $R,\ f:R\rightarrow \mathbb{R}$, 
$f^{-1}(B) \in Suett-L$ f�r jedes $B \in Suett-B$.
Dann $P^X$ Verteilung von $X$, $P^X$ ist W-Ma� auf $R$ und $(R,Suett-L,P^X)$ ist W-Raum f�r reelle Zufallsvarieblen auf $(R,Suett-L,P^X)$.

Der Erwartungswert von $F \cdot X$ existiert genau da,, wenn der Erwartungswert von $f$, aufgefasst als Erwartungswert auf $(R,Suett-L,P^X)$, existiert. Es gilt dann die \textbf{Transformationsformel}
\begin{equation}
	E(f \cdot X) = E_P(f \cdot X) = E_{P^X}(f)
\end{equation}

\section{Spezialf�lle}
\begin{enumerate}
	\item $(R,Suett-L) = (\mathbb{R},Suett-B)$, $X$ habe die Dichte $g$.
		\begin{enumerate}
			\item $f = I_{(-\infty,x]}$
				\begin{equation}
					\begin{split}
						F(x) = P(X \leq x) &= \underbrace{E(I_{(-\infty,x]}\cdot X)}_{= E(f\cdot X)} \\
															 &= E_{P^X}(I_{(-\infty,x]}) \\
															 &= P^X((-\infty,x]) = \int_{-\infty}^{x}{} \\
															 &= \int_{-\infty}^{x}{f(t)g(t)dt}
					\end{split}
				\end{equation}
			\item Die Identit�t \[ E(f \cdot X) = \int_{-\infty}^{x}{f(t)g(t)dt} \]
						gilt f�r ``beliebige'' $f:\mathbb{R}\rightarrow\mathbb{R}$ (mit $f^{-1}{(B)} \in Suett-B$ f�r jedes $B\in Suett-B$, mit existierendem $E(f\cdot X)$).\\
		\end{enumerate}
	\item $X$ habe eine diskrete Verteilung, d.h. es gibt eine abz�hlbare Menge $A \subset R$ mit $P(X\in A)=1,\ (\{x\} \in Suett-L \forall x\in R)$.
	\[ P^X = \sum_{a \in A}{P(X=a)\cdot \delta_a} \]
	F�r $f \geq 0$ ist
	\begin{equation*}
		\begin{split}
		E(f\cdot X) &= E\left( \sum_{a\in A}{f(a)\cdot I(X=a)} + \underbrace{E\left( X \cdot I(X \neq A) \right)}{= 0} \right) \\
								&= \sum_{a\in A}{f(a)\cdot E(I(X=a))} \\
								&= \sum_{a\in A}{f(a)\cdot P(X=a)}
		\end{split}								
	\end{equation*}
	Diese Gleichung ist besonders wichtig, da sie f�r beliebige $f$ gilt mit existierendem $E(f \cdot X)$
\end{enumerate}

\section{Anwendungen}

\subsection{Erwartungswert der Rechteck-Verteilung}

Mit $X \sim Suett-R(a,b)$ gilt \[ g(t) = \begin{cases} \frac{1}{b-a} &,\ a\leq t \leq b \\ b &,\ \text{sonst} \end{cases} \]
Dann ist der Erwartungswert
\begin{equation}
	E(X) = \frac{1}{b-a}\cdot \int_a^b{t\cdot dt} = \frac{a+b}{2}
\end{equation}

\subsection{Erwartungswert der Normalverteilung}

Sei $X \sim N(a,\sigma)$. Dann ist der Erwartungswert\\
\ldots

\renewcommand{\labelenumi}{\theenumi}