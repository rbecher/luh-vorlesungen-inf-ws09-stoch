
$(\Omega,\mathfrak{A},P)$ sei W-Raum. Dann $X_i:\Omega \Rightarrow R_i,\ i=1,\ldots,n \text{Zufallsvariable}$
$X_1,\ldots,X_n$ sind unabh�ngig, wenn gilt
\[ P(X_1\in B_1,\ldots,X_n\in B_n) = P(X_1\in B_1)\cdots P(X_n\in B_n) \]
f�r jede Auswahl von Ereignissen $B_i \in \mathfrak{A}, i=1,\ldots,n$

\ldots

\subsection{Spezialf�lle}
\begin{enumerate}
	\item Die $X_1,\ldots,X_n$ haben je eine diskrete Verteilung.
			  Dann sind 
			  \[ X_1,\ldots,X_n \text{ unabh�ngig } \iff P(X_1=x_1,\ldots,X_n=x_n) = P(X_1=x_1)\cdots P(X_n = x_n)\] 
			  f�r jede Auswahl von Elementen $x_i \in R_i,\ i=1,\ldots,n$.\\
			  (Beachte hier: $\{x\}\in R_i,\ \forall x\in R,\ \forall i\in I$)
	\item Sei $X=(X_1,\ldots,X_n)$ mit den reellen Zufallsvariablen $X_1,\ldots,X_n$.
				Es habe $X$ eine Dichte $f$, das hei�t
				\[ P(X_1 \leq x_1, \ldots, X_n \leq x_n) = \int_{-\infty}^{x_n} \cdots \int_{-\infty}^{x_1}{f(t_1,\ldots,t_n)dt_1,\ldots,dt_n},\ (x_1,\ldots,x_n \in R^n) \]
				Dann hat $X_i$ die Dichte
				\[ f_i(x_i) = \int_{-\infty}^{+\infty} \cdots 	\int_{-\infty}^{+\infty}{f(t_1,\ldots,t_{i-1},x_i,t_{i+1},\ldots,t_n)dt_1,\ldots,dt_n} \]
\end{enumerate}

Gilt
\begin{equation}
	f(x_1,\ldots,x_n) = \prod_{i=1}^{n}{f_i(x_i)},\ (x_1,\ldots,x_n)\in\mathbb{R}^n
\end{equation}
so sind die Zufallsvariablen $X_1,\ldots,X_n$ unabh�ngig.

\subsection{Beispiel}
Ein Einzelexperiment, bei dem ein interessierendes Ereignis $A$ mit Wahrscheinlichkeit $p\in [0,1]$ eintritt, wird $n$-mal unter identischen Versuchsbedingungen ohne gegenseitige Beeinflussung (unabh�ngige Versuchswiederholungen) wiederholt.\\
Beschreibe das Einzelexperiment durch den W-Raum $(\Omega_0,\mathfrak{A}_0,P_0)=(\{0,1\}, \mathfrak{P}(\{0,1\}),P_0)$ mit 
\begin{itemize}
	\item $P_0(\{0\})=1-p$
	\item $P_0(\{1\})=p$
	\item $0 \cong A \text{ tritt nicht ein}$
	\item $1 \cong A \text{ tritt ein}$
\end{itemize}

Beschreibe das Gesamtexperiment durch
\[ (\Omega,\mathfrak{A},P) \text{ mit } \Omega=\{0,1\}^n=\{(\omega_1,\ldots,\omega_n); \omega_i \in \{0,1\}, i=1,\ldots,n\}, \mathfrak{A}=\mathfrak{P}(\Omega) \]
\begin{equation}
	\begin{split}
		P(\{\omega_1,\ldots,\omega_n\}) &= P_0(\{\omega_1\})\cdot P_0(\{\omega_n\}) \\
																		&= p^{\omega_1}\cdot (1-p)^{1-\omega_1}\cdots p^{w_n}\cdot(1-p)^{1-\omega_n} \\
																		&= p^{\omega_1+\ldots+\omega_n}\cdot (1-p)^{n-(\omega_1+\ldots+\omega_n)}, (\omega_1,\ldots,\omega_n) \in \Omega
	\end{split}
\end{equation}

$X_i:\Omega\rightarrow\{0,1\}$ sei definiert durch
\[ X_i(\omega_1,\ldots,\omega_n) = \omega_i, (\omega_1,\ldots,\omega_n)\in\Omega, i=1,\ldots,n \]
Die $X_1,\ldots,X_n$ sind unabh�ngig, denn
\begin{equation}
	\begin{split}
		&P(X_1=\epsilon_1,\ldots,X_n=\epsilon_n) \\
	= &P(\text{w1toninOmega};\ X_1(\omega_1,\ldots,\omega_n)=\epsilon_1,\ldots,X_n(\omega_1,\ldots,\omega_n)=\epsilon_n) \\
	=	& P(\{(\epsilon_1,\ldots,\epsilon_n)\}\\
	=	& p^{\epsilon_1+\ldots+\epsilon_n}\cdot(1-p)^{n-(\epsilon_1+\ldots+\epsilon_n)} \\
	=	& \prod_{i=1}^{n}{p_i^{\epsilon_i}\cdot(1-p)^{1-\epsilon_i}}
	\end{split}
\end{equation}

\begin{equation}
	\begin{split}
		P(X_1=\epsilon_1) &= P(\{(\omega_1,\ldots,\omega_n)\in\Omega; X_1(\omega_1,\ldots,\omega_n)=\epsilon_1\}) \\
											&= P(\{ (\epsilon_1,\omega_2,\ldots,\omega_n); \omega_i\in\{0,1\}, 2\leq i\leq n \}) \\
											&= \sum_{nnn}{P}\ldots
	\end{split}
\end{equation}

\begin{equation}
	\begin{split}
		P(X=k) &= \ldots
	\end{split}
\end{equation}

\section{Definition: Binomialverteilung}

Eine reelle (oder $\{0,1\}$-wertige) Zufallsvariable $X$ hei�t Binomailverteilt mit den Parametern $n\in\mathbb{N}$ und $p\in[0,1]$, wenn gilt
\begin{equation}
	P(X=k) = \binom{n}{k}\cdot p^k \cdot (1-p)^{n-k}, k \in \{0,1,\ldots,n\}
\end{equation}
Schreibweise: \[ X \sim \mathfrak{B}(n,p) \]

\subsection{Anwendung}
Sei $(\Omega,\mathfrak{A},P)$ W-Raum, $A\in \mathfrak{A}$ Ereignis. Dann hei�t $I_A:\Omega \rightarrow \{0,1\}$ \textbf{Indikator} (oder Indikatorvariable) von $A$, definiert durch
\[ I_A(\omega) = \begin{cases}
	1, &\ \omega \in A \\
	0, &\ \omega \notin A.
\end{cases} \]

$I_A \sim \mathfrak{B}(1,p)$ mit $p=P(A)$.\\
Seien $A_1,\ldots,A_n\in \mathfrak{A}$ unabh�ngige Ereignisse mit $p=P(A_i), i=1,\ldots,n.$.\\
Dann gilt $\underbrace{X_1}_{=I_{A_1}},\ldots,\underbrace{X_n}_{=I_{A_n}}$ unabh�ngig und $X_i \sim \mathfrak{B}(1,p)$. $X=X_1+\ldots+X_n \sim \mathfrak{B}(n,p)$.

\subsection{Beispiel}

Eine Urne enth�lt $r$ rote und $s$ schwarze Kugeln. Uns k�nnen verschiedene Modi interessieren.
\begin{enumerate}
	\item Es wird $n$-mal mit Zur�cklegen je eine Kugel gezogen.\\
				Sei $X$ Zufallsvariable und bezeichne die Anzahl der gezogenen roten Kugeln in $n$ Ziehungen.\\
				Dann $X \sim \mathfrak{B}\left(n,\frac{r}{r+s}\right)$
	\item Es wird $n$-mal ohne Zur�cklegen je eine Kugel gezogen mit $n\leq r+s =: a$.\\
				Sei $X$ Zufallsvariable und bezeichne die Anzahl der gezogenen roten Kugeln in $n$ Ziehungen.\\
				Nummeriere die Kugeln in folgender Weise:
				\[\underbrace{1,\ldots,r}_{\text{rot}},\underbrace{r+1,\ldots,\overbrace{r+s}^{=a}}_{\text{schwarz}}\]
				Dann
				\begin{itemize}
					\item $\Omega=\{(\omega_1,\ldots,\omega_n);\ \omega_i\in\{1,\ldots,a\} \text{paarweise verschieden}, 1\leq i \leq n \}$
					\item $\mathfrak{A}=\mathfrak{P}(\Omega)$
					\item $P(A) = \frac{|A|}{|\Omega|}=\frac{|A|}{a\cdot(a-1)\cdot(a-n+1)},\ A\subset\Omega$
					\item \begin{equation}
								\begin{split}
									P(X=k) &= \frac{|\{(\omega_1,\ldots,\omega_n)\in\Omega; |j\in \{1,\ldots,n\}, \omega_j\in\{1,\ldots,r\}|=k\}|}{|\Omega|}\\
												 &= \frac{\binom{n}{k}\cdot r\cdot(r-1)\cdots(r-k+1)\cdot s\cdot(s-1)\cdots(s-(n-k)+1)}{a\cdot(a-1)\cdots(a-n+1)} \\
												 &= \frac{\binom{r}{k}\cdot\binom{a-r}{n-k}}{\binom{a}{n}}, k\in\mathbb{N}, max(0,a-r-n)\leq k \leq min(n,r)
								\end{split}
								\end{equation}
				\end{itemize}
\end{enumerate}

\section{Definition: Hypergeometrische Verteilung}

Eine reelle Zufallsvariable mit der Verteilung
\[ P(X=k) = \frac{\binom{r}{k}\cdot\binom{a-r}{n-k}}{\binom{a}{n}} \]
hei�t hypergeometrisch verteilt mit den Parametern $a,r,n\in\mathbb{N}, r\leq a, n\leq a$ und der Schreibweise
\begin{equation}
	X \sim \mathfrak{H}(a,r,n).
\end{equation}

Das bekannteste Beispiel d�rfte das Zahlenlotto ``6 aus 49'' sein.