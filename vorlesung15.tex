\subsection{Anwendung}

$X_1,X_2,\ldots$ unabh�ngig, jemit derselben Poisson-Verteilung $X_i \sim \mathfrak{P}(\lambda),\ \lambda > 0$.
\begin{equation}
	\sqrt{n}\cdot\left(\frac{1}{n}\cdot\sum_{j=1}^n{X_j} - \lambda\right) \stackrel{\nu}{\rightarrow}N(0,\lambda)
\end{equation}

Sei $f(x)=\sqrt{x}$ f�r $x\geq 0$ f�r differenzierbar in $\lambda>0$ und $f'(\lambda)=\frac{1}{2}\lambda^{-1/2}$.\\
Damit gilt
\begin{equation}
	\sqrt{n}\cdot\left(\sqrt{\frac{1}{n}\cdot\sum_{j=1}^n{X_j}} - \sqrt{\lambda}\right) \stackrel{\nu}{\rightarrow}N(0,\lambda\cdot\frac{1}{4}\cdot\lambda^{-1}) = N(0,\frac{1}{4})
\end{equation}
Allgemein gilt
\begin{equation}
	2\cdot\sqrt{n}\cdot\left(\frac{1}{n}\cdot\sum_{j=1}^n{X_j} - \lambda\right) \stackrel{\nu}{\rightarrow}N(0,1)
\end{equation}

\subsection{Spezialfall}

Sei $\lambda=1$. Dann ist \[S_N := \sum_{j=1}^n{X_j} \sim \mathfrak{P}(n)\]
Es ist dann \[2\cdot\left(\sqrt{S_n}-\sqrt{n}\right) \stackrel{\nu}{\rightarrow}N(0,1)\]
Allgemein soll \[2\cdot\left(\sqrt{X_j}-\sqrt{\lambda}\right) \stackrel{\nu}{\rightarrow}N(0,1),\ \forall \lambda \rightarrow \infty\] gezeigt werden.
Zeige zun�chst \[\frac{X_\lambda-\lambda}{\sqrt{\lambda}} \stackrel{\nu}{\rightarrow}N(0,1)\]

Beweis mit Fourier-Transformierte (ausgelassen).

\subsection{Wurzeltransformation f�r die Poisson-Verteilung}

Es gilt also
\begin{equation}
	\sqrt{\lambda}\cdot\left(\frac{X_\lambda}{\lambda}-1\right) \stackrel{\nu}{\rightarrow}N(0,1)
\end{equation}
Mit dem Fehlerfortpflanzungsgesetz folgt
\begin{equation}
	\sqrt{\lambda}\cdot\left(\sqrt{\frac{X_\lambda}{\lambda}}-\sqrt{1}\right) \stackrel{\nu}{\rightarrow}N(0,\frac{1}{4})
\end{equation}
so dass gilt
\begin{equation}
	\left(\sqrt{X_\lambda}-\sqrt{\lambda}\right) \stackrel{\nu}{\rightarrow}N(0,\frac{1}{4})
\end{equation}
oder alternativ
\begin{equation}
	2\cdot\left(\sqrt{X_\lambda}-\sqrt{\lambda}\right) \stackrel{\nu}{\rightarrow}N(0,1)
\end{equation}
Diese Wurzeltransformation ist eine varianzstabilisierende Transformation f�r die Poissonverteilung.

\subsection{Bemerkung}

Als Erg�nzung zu den lokalen zentralen Grenzwerts�tzen f�r die Binomial-, bzw. Poisson-Verteilung l�sst sich auch ein lokaler zentraler Grenzwertsatz f�r Hypergeometrische Verteilung formulieren:\\
F�r $n\in\mathbb{N}$ sei $X_n\sim\mathfrak{h}(a_n,r_n,n)$ mit $Var(X_n)=n\cdot\frac{r_n}{a_n}\cdot\frac{a_n-r_n}{a_n}\cdot{a_n-n}{a_n-1} \rightarrow \infty,\ (n\rightarrow\infty)$. Dann gilt der "`Zentrale Grenzwertsatz f�r die Hypergeometrsiche Verteilung in lokaler Form"':
\begin{equation}
	P(X_n=k) \sim \frac{1}{\sqrt{2\cdot\pi\cdot Var(X_n)}}\cdot \exp{\left(- \frac{1}{2}\cdot \frac{(k-E(X_n))^2}{Var(X_n)}\right)}
\end{equation}
f�r $|k-E(X_n)| \leq const\ \sqrt{Var(X_n)},\ n\rightarrow\infty$.

Damit gilt f�r $-\infty < a \leq b < +\infty$ der "`Zentrale Grenzwertsatzf�r die Hypergeometrische Verteilung in kumulativer Form"':
\begin{equation}
	P\left(a \leq \frac{X_n-E(X_n)}{\sqrt{Var(X_n)}} \leq b\right) \rightarrow \frac{1}{\sqrt{2\cdot\pi}} \int_a^b{\exp{\left(-\frac{1}{2}\cdot x^2\right)}\cdot dx}
\end{equation}

% Bemerkungen zur Klausur:
% Mit zwei von sechs gel�sten Aufgaben ist die Klausur bestanden (�berhangklausur)
% kombinatorik
% Urnenmodelle
% Erwartungswerte berechnen (3 Techniken: 
% 1) ZVA besitzt Dichte. Dann E(X) = \int_{-\infty}^{+\infty}{f(x)dx}
% 2) diskrete Verteilung. Dann E(X) = \sum_{x\in R} x\cdot P(X=x)
% 3) Summe von Indikatorvariablen (X = \sum_i^n I_A_i). Dann E(X) = \sum_i^n P(A_i)
% 4) Wenn X Nullwertig \in N_0-verteilt. Dann E(X) = \sum_n=0^\infty P(X>n)
% Varianzen und Co-Varianzen
% Cov(X_1,X_2) = E(X_1,X_2) - E(X_1)E(X_2)
% Cov(X_1,X_2) = E(X_1 - E(X_1))(X_2 - E(X_2))
% Verteilungen:
% - Poisson, Binomial, Hypergeometrisch, Negativ-Binomial , Laplace(alle diskret)
% - Normal, Exponential, Rechteck/Gleichverteilung (stetig)
% Ungleichungen
% Cauchy-Schwartz, Chebyshev und Markov
% Chebyshev: P(|X-E(X)|\geq\epsilon) \leq \frac{1}{\epsilon^2}\cdot Var(X)
% Erzeugende Funktionen:
% Sei f_X erz. Fkt. von X, dann
%  E(X) = f_X'(1)
%  Var(f_X''(1)+f_X'(1)-(f_X'(1))^2
%  f_X(t) = \sum P(X=k) t^k

% Unabh�ngigkeit
% X;Y unabh�ngig P(X \leq x, Y \leq y) = P(X \leq x) P(Y\leq <y)
% X N_0 wertig: P(X=k,Y\leq y) = P(X=k) P(Y \leq y)
% E(X Y) = E(X) E(Y), falls X,Y unabh�ngig

% Dichte von Exp(\lambda): f_X(x) = \lambda \exp(-\lambda x) f�r x \geq 0 und 0 sonst