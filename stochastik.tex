%% Basierend auf einer TeXnicCenter-Vorlage von Tino Weinkauf.
%%%%%%%%%%%%%%%%%%%%%%%%%%%%%%%%%%%%%%%%%%%%%%%%%%%%%%%%%%%%%%

%%%%%%%%%%%%%%%%%%%%%%%%%%%%%%%%%%%%%%%%%%%%%%%%%%%%%%%%%%%%%
%% HEADER
%%%%%%%%%%%%%%%%%%%%%%%%%%%%%%%%%%%%%%%%%%%%%%%%%%%%%%%%%%%%%
\documentclass[a4paper,twoside,12pt,notitlepage]{article}
% Alternative Optionen:
%	Papiergr��e: a4paper / a5paper / b5paper / letterpaper / legalpaper / executivepaper
% Duplex: oneside / twoside
% Grundlegende Fontgr��en: 10pt / 11pt / 12pt

%% Deutsche Anpassungen %%%%%%%%%%%%%%%%%%%%%%%%%%%%%%%%%%%%%
\usepackage[ngerman]{babel}
\usepackage[T1]{fontenc}
\usepackage[ansinew]{inputenc}
\usepackage{setspace}

\usepackage{lmodern} %Type1-Schriftart f�r nicht-englische Texte

%% Packages f�r Grafiken & Abbildungen %%%%%%%%%%%%%%%%%%%%%%
%\usepackage{graphicx} %%Zum Laden von Grafiken
%\usepackage{graphics}
%\usepackage{subfig} %%Teilabbildungen in einer Abbildung
%\usepackage{tikz} %%Vektorgrafiken aus LaTeX heraus erstellen
%\usepackage{qtree}

%% Packages f�r Formeln %%%%%%%%%%%%%%%%%%%%%%%%%%%%%%%%%%%%%
%\usepackage{amsmath}
%\usepackage{amsthm}
%\usepackage{amsfonts}

%% Zeilenabstand %%%%%%%%%%%%%%%%%%%%%%%%%%%%%%%%%%%%%%%%%%%%
\usepackage{setspace}
\singlespacing        %% 1-zeilig (Standard)


%% Andere Packages %%%%%%%%%%%%%%%%%%%%%%%%%%%%%%%%%%%%%%%%%%
\usepackage{a4wide} %%Kleinere Seitenr�nder = mehr Text pro Zeile.
\usepackage{fancyhdr} %%Fancy Kopf- und Fu�zeilen
\renewcommand{\headheight}{15pt}
\usepackage{fancybox}

%\usepackage{longtable} %%F�r Tabellen, die eine Seite �berschreiten
\usepackage{ifpdf}
\ifpdf
	\usepackage[hyperindex,colorlinks,bookmarks,urlcolor=blue]{hyperref}
\else
	\usepackage{url}
\fi

%% FancyHeader Definitionen %%%%%%%%%%%%%%%%%%%%%%%%%%%%%%%%%%%%%%%%%%

\lhead{Stochastik A, WS 2009}
%\chead{}
\rhead{Mitschrift}
\cfoot{\thepage}

%% Basierend auf einer TeXnicCenter-Vorlage von Tino Weinkauf.

%%%%%%%%%%%%%%%%%%%%%%%%%%%%%%%%%%%%%%%%%%%%%%%%%%%%%%%%%%%%%%


%%%%%%%%%%%%%%%%%%%%%%%%%%%%%%%%%%%%%%%%%%%%%%%%%%%%%%%%%%%%%

%% OPTIONEN

%%%%%%%%%%%%%%%%%%%%%%%%%%%%%%%%%%%%%%%%%%%%%%%%%%%%%%%%%%%%%

%%

%% ACHTUNG: Sie ben�tigen ein Hauptdokument, um diese Datei

%%          benutzen zu k�nnen. Verwenden Sie im Hauptdokument

%%          den Befehl "\input{dateiname}", um diese

%%          Datei einzubinden.

%%



%%%%%%%%%%%%%%%%%%%%%%%%%%%%%%%%%%%%%%%%%%%%%%%%%%%%%%%%%%%%%

%% ABK�RZUNGEN

%%%%%%%%%%%%%%%%%%%%%%%%%%%%%%%%%%%%%%%%%%%%%%%%%%%%%%%%%%%%%


%%Definitionen f�r Abbildung-Referenzen

\newcommand{\abb}[1]{(Abbildung \ref{#1})}

\newcommand{\ABB}[1]{Abbildung \ref{#1}}


%%Definitionen f�r Tabellen-Referenzen

\newcommand{\tab}[1]{(Tabelle \ref{#1})}

\newcommand{\TAB}[1]{Tabelle \ref{#1}}


%%Definitionen f�r Seiten-Referenzen

\newcommand{\seite}[1]{(Seite \pageref{#1})}

\newcommand{\SEITE}[1]{Seite \pageref{#1}}


%%Definitionen f�r Code

\newcommand{\precode}[1]{\textbf{\footnotesize #1}}

\newcommand{\code}[1]{\texttt{\footnotesize #1}}


%% Basierend auf einer TeXnicCenter-Vorlage von Tino Weinkauf.

%%%%%%%%%%%%%%%%%%%%%%%%%%%%%%%%%%%%%%%%%%%%%%%%%%%%%%%%%%%%%%


%%%%%%%%%%%%%%%%%%%%%%%%%%%%%%%%%%%%%%%%%%%%%%%%%%%%%%%%%%%%%

%% OPTIONEN

%%%%%%%%%%%%%%%%%%%%%%%%%%%%%%%%%%%%%%%%%%%%%%%%%%%%%%%%%%%%%

%%

%% ACHTUNG: Sie ben�tigen ein Hauptdokument, um diese Datei

%%          benutzen zu k�nnen. Verwenden Sie im Hauptdokument

%%          den Befehl "\input{dateiname}", um diese

%%          Datei einzubinden.

%%


%%%%%%%%%%%%%%%%%%%%%%%%%%%%%%%%%%%%%%%%%%%%%%%%%%%%%%%%%%%%%

%% OPTIONEN F�R ABST�NDE

%%%%%%%%%%%%%%%%%%%%%%%%%%%%%%%%%%%%%%%%%%%%%%%%%%%%%%%%%%%%%


%%Abstand zwischen den Abs�tzen: halbe H�he vom kleinen x

\setlength{\parskip}{0.5ex}


%%Einzug am Anfang eines Absatzes: auf Null setzen

%\setlength{\parindent}{0ex}


%%Zeilenabstand: 1.5 fach

%% ==> Erw�gen Sie, anstelle dieses Kommandos das Paket 'setspace' zu verwenden.

%\linespread{1.5}



%%%%%%%%%%%%%%%%%%%%%%%%%%%%%%%%%%%%%%%%%%%%%%%%%%%%%%%%%%%%%

%% OPTIONEN F�R KOPF- UND FUSSZEILEN

%%%%%%%%%%%%%%%%%%%%%%%%%%%%%%%%%%%%%%%%%%%%%%%%%%%%%%%%%%%%%

%%Beispiel f�r recht nette Kopf- und Fu�zeilen

%% ==> Nutzen Sie '\usepackage{fancyhdr}' und '\pagestyle{fancy}'

%% ==> im Hauptdokument, um diese zu benutzen.

%\pagestyle{fancy}

\renewcommand{\chaptermark}[1]{\markboth{#1}{}}

\renewcommand{\sectionmark}[1]{\markright{\thesection\ #1}}

\fancyhf{}

\fancyhead[LE,RO]{\thepage}

\fancyhead[LO]{\rightmark}

\fancyhead[RE]{\leftmark}

\fancypagestyle{plain}{%

    \fancyhead{}

    \renewcommand{\headrulewidth}{0pt}

}



%% Basierend auf einer TeXnicCenter-Vorlage von Tino Weinkauf.

%%%%%%%%%%%%%%%%%%%%%%%%%%%%%%%%%%%%%%%%%%%%%%%%%%%%%%%%%%%%%%


%%%%%%%%%%%%%%%%%%%%%%%%%%%%%%%%%%%%%%%%%%%%%%%%%%%%%%%%%%%%%

%% PDF-Informationen

%%%%%%%%%%%%%%%%%%%%%%%%%%%%%%%%%%%%%%%%%%%%%%%%%%%%%%%%%%%%%

%%

%% ACHTUNG: Sie ben�tigen ein Hauptdokument, um diese Datei

%%          benutzen zu k�nnen. Verwenden Sie im Hauptdokument

%%          den Befehl "\input{dateiname}", um diese

%%          Datei einzubinden.

%%


\pdfinfo{                               % Zusatzinformationen in PDF-Datei;

                                        % alle Werte sind optional.

    /Author (Ronald Becher)

    /CreationDate (D:20090928120000)    % Datum der Erstellung

                                        % (D:JJJJMMTThhmmss)

                                        % JJJJ  Jahr

                                        % MM    Monat

                                        % TT    Tag

                                        % hh    Stunden

                                        % mm    Minuten

                                        % ss    Sekunden

                                        %

                                        % Standard: Das aktuelle Datum

                                        %

    /ModDate (\date)         % Datum der letzten Modifikation

    /Creator (TeX && TXC)               % Standard: "TeX"

    /Producer (pdfTeX)                  % Standard: "pdfTeX" + pdftex version

    /Title (Vorlesungsmitschrift Stochastik Wintersemester 2009)

    /Subject (Thema Ihres Dokumentes)

    /Keywords (stochastik, mathematik, vorlesung, mitschrift)

}




\title{Stochastik Wintersemester 2009\\Leibniz Universit�t Hannover\\Vorlesungsmitschrift}

\author{Dozent: \href{mailto:lbaring@stochastik.uni-hannover.de}{Prof. Dr. L. Baringhaus}\\ \vspace{1em} \\
Mitschrift von\\
\href{mailto:rb@ronald-becher.com}{Ronald Becher}}

\begin{document}
\pagestyle{empty}
\maketitle
\begin{abstract}
	Diese Mitschrift wird erstellt im Zuge der Vorlesung \href{http://www.stochastik.uni-hannover.de/bachws09/index.html}{Stochastik A} im Wintersemester 2009 an der \href{http://www.uni-hannover.de}{Leibniz Universit�t Hannover}. Obwohl mit gro\ss er Sorgfalt geschrieben, werden sich sicherlich Fehler einschleichen. Diese bitte ich zu melden, damit sie korrigiert werden k�nnen. Ihr k�nnt euch dazu an jeden der genannten Autoren (mit Ausnahme des Dozenten) wenden.
	
	Dieses Skript wird prim�r �ber \href{http://github.com/rbecher/luh-vorlesungen-inf-ws09-stoch}{Github} verteilt und "`gepflegt"'. Dort kann man es auch forken, verbessern und dann (idealerweise) ein "`Pull Request"' losschicken. Siehe auch  \href{http://github.com/guides}{Github Guides}. Auch wenn ihr gute Grafiken zur Verdeutlichung beitragen k�nnt, d�rft ihr diese gerne schicken oder selbst einf�gen (am besten als \LaTeX -geeignete Datei (Tikz, SVG, PNG, \dots)).
	
	\textbf{Hinweis}: Es wird o.B.d.A. davon abgeraten, die Vorlesung zu vers�umen, nur weil eine Mitschrift angefertigt wird! Selbiges gilt f�r den �bungsbetrieb! Stochastik besteht ihr nur, wenn ihr auch zu der Veranstaltung geht.
\end{abstract}

\newpage % \newpage works far better than \pagebreak

\tableofcontents

%Part: F�r gro�e Themenbl�cke (z.B. Einf�hrung, Kombinatorik, etc)
%Section: F�r Abschnitte (z.B. Mengen, von Mengen zu anderen Mengen, etc)
%Subsection: Beweise, S�tze, Lemmata

\newpage

\pagestyle{fancy}

% Here comes content

\part{Einf�hrung}
\label{sec:Einf�hrung}

\section{Noch leer}
\label{sec:Mengen}

\subsection{Die Formel von Sylvester-Poincar�}
\label{sec:Siebformel}

\par Und hier kommt schon die schwierigste Formel in der ganzen Vorlesung:
\begin{displaymath}
\mathsf{P}\left(\bigcup_{i=1}^nA_i\right) = \sum_{k=1}^n(-1)^{k+1}\!\!\sum_{I\subseteq\{1,\dots,n\},\atop |I|=k}\!\!\!\!\mathsf{P}\left(\bigcap_{i\in I}A_i\right)
\end{displaymath}

\subsection{Spielereien mit fancybox}
\label{sec:fancybox}

\begin{displaymath}
	x + y = \fbox{$\Omega$}
\end{displaymath}

\begin{equation}
\fbox{$\displaystyle
\int_{\Omega_0} \zeta(\omega) d\omega
\geq \bar{r}$}
\end{equation}

\newlength{\mylength}
\[
\setlength{\fboxsep}{15pt}
\setlength{\mylength}{\linewidth}
\addtolength{\mylength}{-2\fboxsep}
\addtolength{\mylength}{-2\fboxrule}
\fbox{%
\parbox{\mylength}{
\setlength{\abovedisplayskip}{0pt}
\setlength{\belowdisplayskip}{0pt}
\begin{equation}
x + y = z
\end{equation}}}
\]

\newenvironment{FramedEqn}%
{\setlength{\fboxsep}{15pt}
\setlength{\mylength}{\linewidth}%
\addtolength{\mylength}{-2\fboxsep}%
\addtolength{\mylength}{-2\fboxrule}%
\Sbox
\minipage{\mylength}%
\setlength{\abovedisplayskip}{0pt}%
\setlength{\belowdisplayskip}{0pt}%
\equation}%
{\endequation\endminipage\endSbox
\[\fbox{\TheSbox}\]}

\begin{FramedEqn}
\Rightarrow P\sim\xi(P_\gamma)- \frac{1}{3}
\end{FramedEqn}

\fbox{%
\begin{Beqnarray*}
x & = & y\\
y & > & x \\
\int_4^5 f(x)dx & = & \sum_{i\in F} x_i
\end{Beqnarray*}}

\end{document}